\chapter{IT-Konzept}

\section{Logische Einheiten}

In den nächsten Abschnitten wird auf die verschieden Logischen Einheiten eingegangen. Logische Einheiten
sind Komponenten oder auch Klassen, die eine klar diffenrzierbare Aufgabe innerhalb der Applikation haben.

\subsection{App, main() und View}
Die Klasse App ist bei der Erstellung eines neuen
Projekts von Netbeans\footnote{NetBeans IDE 7.0 (Build 201104080000)} erstellt worden.
Die Klasse enthält die \texttt{main()}-Methode, welche zu dem Starten einer jeden Java-Applikation benötigt wird.
Die Klasse App enthält eine Methode, welche die Klasse View instanziert.

\definecolor{bg}{rgb}{0.95,0.95,0.95}
\begin{listing}[H]
    \begin{minted}[bgcolor=bg,linenos=true,fontsize=\footnotesize]{java}
 /**
 * At startup create and show the 
 * main frame of the application.
 */
 @Override protected void startup() {
     show(new View(this));
 }    
    \end{minted}
    \caption{Methode startup ( App.java Z.18-20 )}
    \label{code:app:startup}
\end{listing}

\subsubsection{Der View}
In dieser Klasse befinden sich alle Methoden, die zur Generierung der einzelen Steuerelemente dienen.
Ferner werden alle Balkendiagramme und deren zugehörige Algorithmen instanziert. Zudem werden alle Eventhandler angelegt,
mit denen die Anwendereingaben delegiert werden können.

\subsection{Die RunnableSortingCollection}
\texttt{RunnableSortingCollection}\footnote{"`There are only two hard things in Computer Science: cache invalidation and naming things"' \newline Phil Karlton} 
Eine Instanz dieser Klasse beinhaltet alle Sortieralgorithmen, sofern programmatisch hinzugefügt, und delegiert
eventuelle Steuerbefehle. Ein Beispiel ist das Starten des Sortiervorgangs. Die Klasse dient der 
Kommunikation zwischen Nutzeroberfläche, beziehungsweise Nutzer, und den Algorithmen.

\subsection{Der SynchronizedSorter}
Wie zu Anfang beschrieben braucht es Logik um zu gewährleisten, dass der Ablauf des Sortierens paralell und synchron über den ganzen
"`Existenzzeitraum"' der Applikation besteht. Erreicht wird dieses Verhalten durch die Nutzung von Native-Threads und der Tatsache 
dass seit Java 5 Threads ein Interkommunikationsmodel besitzen, die 
sich per Java programmieren lassen\footnote{siehe http://download.oracle.com/javase/tutorial/essential/concurrency/}.

\subsection{Der VisualFeedbackSorter}
Eine der wichtigesten Komponenten der Applikation ist der \texttt{VisualFeedbackSorter}, welcher für die visuelle Repränstation 
der internen Sortiervogänge zuständigt ist. Wie auch der \texttt{SynchronizedSorter} bedient sich der \texttt{VisualFeedbackSorter} dem Decorator-Pattern (siehe  
\ref{sec:decoratorpattern}) um seine Funktionalität der eigentlichen Algorithmusinstanz voranzustellen.

\subsection{Die Klasse \texttt{AbstractSortingMechanics} und das Interface \texttt{Sorter}}
\munepsfig[scale=0.7]{abstractsortingmechanics}{Klassenkarte von AbstractSortingMechanics.java und Sorter.java}
Wie in Abb. \ref{fig:abstractsortingmechanics} dargestellt sind das Interface \texttt{Sorter} und die Klasse \newline \texttt{AbstractSortingMechanics} die Basis 
für den Decorator-Pattern. Zugleich bilden diese beiden Logik Einheiten die Basis für alle Klassen, 
die sich wie ein Algorithmus verhalten sollen (vgl. Abb. \ref{fig:classdiagramm}). 
\newpage

\section{Erläuterung wichtiger Entwurfsmuster}
Entwurfsmuster benennen, abstrahieren und indentifiziern die Kernaspekte einer wiederkehrenden Herrangehensweise und schaffen somit wiederverwendbare
objektorientierte Lösungen \parencite[vgl.][S.3]{designpatterns}.   

\subsection{Decorator-Pattern}
\label{sec:decoratorpattern}
\begin{munquote}[\parencite{decoratorpattern}]
Mit dem Decorator Pattern, in der deutschen Übersetzung naheliegenderweise Dekorierer genannt, 
lässt sich ein Objekt dynamisch um Fähigkeiten, auch Zuständigkeiten genannt, erweitern. 
Anstatt Unterklassen zu bilden und eine Klasse damit um Fähigkeiten bzw. Verhalten zu erweitern, lässt sich mit dem Einsatz des Decorator Patterns 
die Erzeugung von Unterklassen vermeiden
\end{munquote}

\subsubsection{AbstractSortingDecorator}
Diese Klasse bildet die Grundlage für den Decorator-Pattern und wird an \newline \texttt{VisualFeedbackSorter} und \texttt{SynchronizedSorter} weitervererbt.
Wird einen dieser Dekoratoren instanziert und das zu dekorierende Objekt an ihn übergeben, so verhält der Dekorator nach außen hin 
genau so wie das übergebene Objekt. Darin liegt die Stärke diese Entwurfsmusters welches sich in dem Konstruktor von \texttt{View} zeigt.
\newpage
\definecolor{bg}{rgb}{0.95,0.95,0.95}
\begin{listing}
    \begin{minted}[bgcolor=bg,fontsize=\footnotesize,gobble=7,linenos=true]{java}
        Sorter bubblesort    = 
                new SynchronizedSorter(new VisualFeedbackSorter(new Bubblesort()));
        Sorter heapsort      = 
                new SynchronizedSorter(new VisualFeedbackSorter(new Heapsort()));
        Sorter insertionsort = 
                new SynchronizedSorter(new VisualFeedbackSorter(new Insertionsort()));
        Sorter mergesort     = 
                new SynchronizedSorter(new VisualFeedbackSorter(new Mergesort()));    
    \end{minted}
    \caption{Instanzierung und Dekorierung aller Algorithmen (View.java Z.69-73)}
    \label{code:app:startup}
\end{listing}

Nach wie vor verhalten sich alle Algorithmen nach außen hin wie ein \texttt{Sorter}, nur sind jetzt alle Methoden dekoriert. Wird nun von 
\texttt{bubblesort} eine Method aufrufen, so wird sie von \texttt{SynchronizedSorter} über \texttt{VisualFeedbackSorter} bis zu dem eigentlichen \texttt{Bubblesort}
Algorithmus durchgereicht werden. Während dieser Prozedur kann jeder Dekorierer auch eigenen Code ausführen.
Entscheident ist, dass \texttt{SynchronizedSorter} und auch \texttt{VisualFeedbackSorter} in den Konstruktoren den übergebenen Algorithmus in 
einem dafür vorgesehenen Feld speichen.
\definecolor{bg}{rgb}{0.95,0.95,0.95}
\begin{listing}
    \begin{minted}[bgcolor=bg,linenos=true,fontsize=\footnotesize]{java}
 public abstract class AbstractSortingDecorator 
   extends AbstractSortingMechanics {
     protected final Sorter algorithm;

     public AbstractSortingDecorator(Sorter algorithm) {
         this.algorithm = algorithm;
     }
 } 
    \end{minted}
    \caption{ Konstruktor (AbstractSortingDecorator.java Z.11-29)}
    \label{code:app:startup}
\end{listing}

\noindent Außerdem müssen die erbenden Klassen diese Feld zur "`Redelegation"' der Methode benutzen, damit der Decorator-Pattern richtig
implementiert ist (siehe Quellcode \ref{code:ss:swap} Z.17).

\definecolor{bg}{rgb}{0.95,0.95,0.95}
\begin{listing}[H]
    \begin{minted}[bgcolor=bg,linenos=true,fontsize=\footnotesize]{java}
public class SynchronizedSorter extends AbstractSortingDecorator {

    public SynchronizedSorter(Sorter algorithm) {
        super(algorithm);
    }

    @Override
    public void swap(int m, int n) {
        synchronized (super.algorithm) {
            try {
                super.algorithm.wait();
            } catch (InterruptedException ex) {
                Logger.getLogger(SynchronizedSorter.class.getName())
            .log(Level.SEVERE, null, ex);
            }
        }
        algorithm.swap(m, n);
    }
    \end{minted}
    \caption{ Beispiel einer Dekoration ( SynchronizedSorter.java Z.20-36. )}
    \label{code:ss:swap}
\end{listing}

\section{Erläuterung besondere Implementationen}
\subsection{Die Klasse \texttt{SynchronizedSorter}}

\begin{wrapfigure}{r}{0.4\textwidth}
\vspace{-40pt}
  \begin{center}
    \muneps[scale=0.6]{synchronizedsorter}{Klassenkarte}
  \end{center}
  \vspace{-20pt}
  \vspace{-10pt}
\end{wrapfigure}
Wie in Quellcode \ref{code:ss:swap} und Abb. \ref{fig:synchronizedsorter} bereits gezeigt erbt die Klasse von \texttt{AbstractSortingDecorator} 
und dieser wiederum von \texttt{AbstractSortingMechanics}. 
Die Methoden \texttt{swap}, \texttt{compare}, \texttt{assign}, \newline \texttt{getProblemValueAt} werden mit einer 
speziellen Direktive implementiert(siehe Quellcode \ref{code:ss:swap} Z.9-16). Das hat zur Folge,
dass immer wenn eine dieser Methoden aufgerufen wird, der laufende Thread in einen wartendend Zustand versetzt wird
(siehe Quellcode \ref{code:ss:swap} Z.11). Dabei spielt das Objekt auf dem \texttt{wait()} ausgeführt wird keine Rolle. Der Thread wartet 
bis auf dem selben Objekt an einer anderen Stelle \texttt{notify()} ausgeführt wird. Durch diesen Aufbau ist der Ausführungsfluss zur Sortierung synchronisiert.

\subsection{Die Klasse \texttt{VisualFeedbackSorter}}
Diese Klasse baut zum Start der Applikation mit Hilfe der Charting-Bibliothek \newline \texttt{jChart2D}, 
das Balkendiagramm für den übergebenen Algorithmus auf und speichert
die Instanz in einem Feld. Gleichzeitig fügt es die Balken in den View ein. Wird eine dekorierte Methode
aufgerufen, so wird das Chart angewießen die entsprechenden Balken, zum Beispiel bei einer dekorierten \texttt{compare} Methode, mit einer Farbe zu hinterlegen.
Anschließend wird der Aufruf weitergereicht.

\subsection{Die Klasse \texttt{Surveyor}}
Die Klasse Surveyor kann die Sortiertheit eines Integer Arrays im Rahmen von 
0.0 bis 1.0 berechnen. Dabei wird die Distanz eine Elements zu seiner eigentlichen Position bestimmt. Diese Entfernungen werden aufsummiert 
und durch die maximal möglich Gesamtdistanz geteilt. Bei einer länge von $n$ ergibt dies eine maximale Entfernung von $n^2$.
Die Methodik wird für die Auswertung am Ende eines jeden Sortiervorgangs benutzt.

\section{Klassendiagramm}
\munlepsfig[scale=0.74]{classdiagramm}{Klassendiagramm aller wichtigen Komponenten}
In dem gezeigten Klassendiagramm befinden sich alle Klassen und Interfaces, die in der Applikation verwendet werden.
Die Klassen die keinen Bezug zur Sortierstruktur haben und für die Verdeutlichung der Zusammenhänge keine signifikante Rolle spielen sind nicht enthalten.
