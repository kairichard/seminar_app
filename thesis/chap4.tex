\chapter{Refelexion}
\section*{Visualisierung}
Bei der Betrachtung das Endergebnisses kann mann feststellen das dem Anwender leider nicht so einfach klar wird, wie die einzelnen 
Algorithmen tatsächlich funktioniern. Der Ansatz den ich gewählt habe eignet sich aus folgenden Gründen nicht dazu: 
Ich habe angenommen das es reichen würde die einzelen Operation, die ein Algorithmus im Regelfall durchführen kann zu isolieren und dann
mit der Visualisierungslogik zu dekorieren. Daher ist der jeder Algorithmus ist ganz normal implmentiert, ohne spezielle Zusatzlogik.
Die bedeutet im Umkehrschluss auch, dass mit der gegebenen Infrastruktur von Klassen und Interfaces fast jeder Algorithmus visualisiert
werden kann. 

Für sich genommen ist diese Herrangehensweise, denke ich, legitim und auch
für simple Algorithme durchführbar. Jedoch das Ergebnis von der optischen Seite betrachtet nicht zufriedenstellend, da die 
Visualisierung der komplexeren Algorithmen eher wilkürlich wirken.

Nachträglich habe ich deswegen versucht weitere sinnvolle Teilvisualisierungen bei den komplexeren Algorithmen hinzuzufügen. Dennoch
bleibt der Punkt, der eher geringen Anschaulichkeit bestehen, da man nur versteht wie der Algorithmus funktioniert wenn der Source-Code dafür
gleichzeitg betrachtet wird.
Und mit den optischen Herrvorhebungen verknüpft ist. Diese Verfahren habe ich
verwendet um Passagen dieser Seminararbeit zu anschaulicher zu formulieren. Inbesondere bei dem Quicksort-Algorithmus hat mir die Visualisierung
gut geholfen. Jedoch kann ich den Source lesen, was mann nicht von allen anderen behaupten kann. 

\subsection*{Ein besser Ansatz}
Den Fehler, wenn man überhaupt von einem Fehler sprechen kann, liegt klar in der Progammtechnischen orientierten 
Herrangehensweise. Davon ausgehen habe ich Strukturen programmiert, die es einfach machen jeden Algorithmus zu visualisieren.
Die Vorüberlegung, einzelene Operationsschritte in einem größern Zusammenhang und im Zusammenwirkung mit anderen Operationsschritten sinnvoll 
visualisieren, hätte eventuell ein anschaulichers Ergebnis zur Folge. 

