\chapter{Konzepte}
\label{chap:concepts}
\section{ Grobkonzept }
\subsection{Was macht das Programm ?}

Das im im weitern verlauf dieses Textes beschrieben Programm soll anhand eingänglicher
Visualierungen die Funktionsweise von 5 unterschiedlichen Sortieralgorithmen
darstellen. Außerdem soll dem Anwender bei der Betrachtung klar werden welcher Algorithmus 
sich für welche Problemstellung am besten eignet. Erreicht wird dieser Eindruck durch graphische Auswertung der
der durchgeführten Operationen.

\section{ Feinkonzept }
\subsection{ Ablauf des Programmes oder auch die User-Experience }
Definition des Begriffs User-Experience:
\begin{munquote}[\parencite{wikiux}]
   Der Begriff User Experience (Abkürzung UX, deutsch wörtlich: Nutzererfahrung, besser: Nutzererlebnis oder Nutzungserlebnis - alternativ wird auch häufig vom Anwendererlebnis gesprochen) 
  umschreibt alle Aspekte der Erfahrungen eines Nutzers bei der Interaktion mit einem Produkt, Dienst, Umgebung oder Einrichtung. 
  Dazu zählen auch Software und IT-Systeme. Meistens wird im Zusammenhang mit der Gestaltung von Websites von User Experience gesprochen, der Begriff umschließt jedoch tatsächlich das volle Spektrum an Interaktionsmöglichkeiten.
\end{munquote}
