\chapter{Grobkonzept}
\label{chap:concepts}
\section{Was macht das Programm ?}

Das Java-Programm soll die Funktionsweise von fünf unterschiedllichen Sortieralgorithmen\footnote{
 Bubble sort(Type: Exchange sort) | 
 Quicksort(Type: Exchange sort) | 
 Heapsort(Type: Selection sort) | 
 Insertion sort(Type: Insertion sort) | 
 Merge sort(Type: Merge sort) | 
 } mit Hilfe von Diagrammen anschaulich darstellen. 
 Zur Verdeutlichung der Vorgänge wird jedem Algorithmus ein Balken-Diagramme zugewiesen, in dem jeder Zahlenwert
einem Balken entspricht.
Vergleicht der Algorithmus nun zwei Zahlenwerte miteinander, so werden die korrespondierenden 
Balken farblich hervorgehoben. Entscheidet der Algorithmus das einer der Balken 
vorschoben werden muss um alle Zahlenwerte in eine korrekte, aufsteigende Reihenfolge zu bringen,
so wird auch diese Operation im Balken-Diagramm durch farbliche Hervorhebung visualisiert.
\paragraph*
\noindent Um die fünf dargestellten Algorithmen besser vergleichen
zu können, laufen alle Operationen synchronisiert und parallel ab.
Zudem kann jeder Algorithmus pro Zeiteinheit nur eine Änderung an den Zahlenwerten durchführen. Diese Restriktion stellt sicher, dass
die Ergebnisse reproduzierbar bleiben. 
Zusätzlich hat der Benutzer die Möglichkeit spezielle Zahlenwert/Balken-Kombinationen 
auszuwählen um die Schwächen und Stärken einzelner Algorithmen deutlich sichtbarer zu gestalten.
Möglich Kombinationen sind zum Beispiel:
\begin{itemize}
    \item Zahlenwerte, die in mathematisch umgekehrter Reihenfolge vorliegen
    \item Zahlenreihen, in denen es nur eine geringe Anzahl an Zahlenwerten an der "`falschen"' Stelle gibt
    \item Zahlenreihen, bei denen es mehrere gleiche Zahlenwerte gibt
\end{itemize}

\paragraph*
\noindent Wichtig ist vorallem, dass dem Anwender bei der Betrachtung klar 
wird welcher Algorithmus sich für welche Problemstellung am besten eignet. 
Erreicht wird dies durch ein Liniendiagramm welches am Ende der Laufzeit eines jedes Algorithmus  
über das zugewiesene Balken-Diagramm gelegt wird. Durch den Vergleich der Linien ist ersichtlich, wie sich der Grad der "Sortiertheit" 
bei den verschiedenen 
Algorithmen über die Zeit entwickelt. Die Anzahl der durchgeführten Operationen ist unterhalb der Liniendiagramme angegeben.

