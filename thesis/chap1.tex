\chapter{Konzepte}
\label{chap:concepts}
\section{Grobkonzept}
\subsection{Was macht das Programm ?}

Das Java-Programm soll anhand von
Visualisierungen die Funktionsweise von fünf\footnote{
 Bubble sort(Type: Exchange sort) | 
 Quicksort(Type: Exchange sort) | 
 Heapsort(Type: Selection sort) | 
 Insertion sort(Type: Insertion sort) | 
 Merge sort(Type: Merge sort) | 
 } unterschiedlichen Sortieralgorithmen
darstellen. Zur Verdeutlichung der Vorgänge wird jedem Algorithmus ein Balken-Diagramme zugewiesen, bei dem jeder Balken
einem Zahlenwert entspricht, 
vergleicht der Algorithmus nun zwei Zahlen werden die korrespondierende 
Balken farblich hervorgehoben. Entscheidet der Algorithmus das einer der Balken 
vorschoben werden muss um alle Zahlenwerte in eine korrekte, aufsteigende Reihenfolge zu bringen
so wird auch diese Operation im Balken-Diagramm durch einen Pfeil vom Ursprungsort zum 
Zielort visualisiert. \newpage

Um die dargestellten Algorithmen noch besser
verstehen zu können, laufen alle Operation synchronisiert und parallel ab, das heißt
jeder Algorithmus kann pro Zeit nur eine Änderung an den Zahlenwerten durchführen. Diese Restriktion stellt sicher das
die Ergebnisse reproduzierbar und vergleichbar bleiben. 
Dazu hat der Benutzer noch die Möglichkeit spezielle Zahlenwert/Balken Kombinationen 
auszuwählen um schwächen und stärken einzelner Algorithmen besser sichtbar zu machen.
Denkbar wären zum Beispiel Kombinationen die in umgekehrter Reihenfolge vorliegen oder welche bei denen
nur wenige Werte an der 'falschen' stelle sind oder auch solche bei denen es viele gleiche Werte gibt.

Wichtig ist vorallem das dem Anwender bei der Betrachtung klar 
wird welcher Algorithmus sich für welche Problemstellung am besten eignet. 
Erreicht wird das durch ein Ranking welches am Ende der Laufzeit jedes Algorithmus halbtransparent 
über das zugewiesen Balken-Diagramme gelegt wird. Auf dem Transparent werden der Rang, 
sprich wie lange der Algorithmus im vergleich zu den anderen gebraucht hat, 
sowie die Anzahl durchgeführter Operationen zu finden sein.

\newpage

\section{Fachkonzept}
\subsection{Ablauf des Programmes oder auch die User-Experience}
Definition des Begriffs User-Experience:
\begin{munquote}[\parencite{wikiux}]
   Der Begriff User Experience (Abkürzung UX, deutsch wörtlich: Nutzererfahrung, besser: Nutzererlebnis oder Nutzungserlebnis - alternativ wird auch häufig vom Anwendererlebnis gesprochen) 
  umschreibt alle Aspekte der Erfahrungen eines Nutzers bei der Interaktion mit einem Produkt, Dienst, Umgebung oder Einrichtung. 
  Dazu zählen auch Software und IT-Systeme. Meistens wird im Zusammenhang mit der Gestaltung von Websites von User Experience gesprochen, der Begriff umschließt jedoch tatsächlich das volle Spektrum an Interaktionsmöglichkeiten.
\end{munquote}
